\chapter*{Sommario}
\addcontentsline{toc}{chapter}{Sommario}
%problema



%contesto
La \textbf{quarta rivoluzione industriale} ha profondamente cambiato il modo di produrre in fabbrica, grazie alla convergenza di diversi elementi. La diffusione dei \textbf{dispositivi IoT} nelle fabbriche ha permesso la generazione di una grande quantità di dati provenienti dai processi produttivi. Il \textbf{cloud computing} ha fornito l'infrastruttura necessaria per gestirne l'elaborazione. Infine, lo sviluppo dell'intelligenza artificiale, specialmente nel campo dell'\textbf{analisi delle immagini}, ha permesso di estrarre informazioni sempre più accurate, automatizzando operazioni come il controllo qualità dei prodotti, ma soprattutto il monitoraggio dei dispositivi di protezione individuale (DPI).


L'obiettivo della tesi è quello di mostrare nella pratica l'utilizzo di quanto appena presentato, attraverso la generazione di un \textbf{prototipo} per il controllo del corretto indossamento dei dispositivi di sicurezza. In particolare, il sistema deve riuscire ad integrare i flussi video RTSP, provenienti dalle telecamere esistenti nell'impianto, e i messaggi, provenienti dai sensori che indossano gli impiegati, nella piattaforma cloud di \textbf{Amazon Web Services (AWS)}, sfruttando i servizi gestiti offerti dal provider. 

%contesto
%tecnologie

%sistema
%risultati