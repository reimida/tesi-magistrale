\chapter*{Sommario}
\addcontentsline{toc}{chapter}{Sommario}

\vspace{1cm}
%problema
L'incolumità sul luogo di lavoro ancora oggi impatta sulla vita dei lavoratori, delle aziende ed in generale sulla società, sia in termini umani che produttivi. Dal punto di vista normativo è stata predisposta tutta una infrastruttura legale su questo tema, ma le statistiche e la cronaca dimostrano una risposta non sufficiente al problema.  
\vspace{0.7cm}


%contesto
\noindent La \textbf{quarta rivoluzione industriale} ha profondamente cambiato il modo di produrre in fabbrica, grazie alla convergenza di diversi elementi. La diffusione dei \textbf{dispositivi IoT} nell' industria ha permesso la generazione di una grande quantità di dati provenienti dai processi produttivi. Il \textbf{cloud computing} ha fornito l'infrastruttura necessaria per gestirne l'elaborazione. Infine, lo sviluppo dell'intelligenza artificiale, specialmente nel campo dell'\textbf{analisi delle immagini}, ha permesso di estrarre informazioni sempre più accurate, automatizzando operazioni come il controllo qualità dei prodotti, ma soprattutto il monitoraggio dei dispositivi di protezione individuale (DPI).
\vspace{0.7cm}

\noindent L'obiettivo della tesi è quello di mostrare nella pratica l'utilizzo di quanto appena presentato, attraverso la generazione di un \textbf{prototipo} per il controllo del corretto impiego dei dispositivi di sicurezza. In particolare, il sistema deve riuscire ad integrare i flussi video RTSP, provenienti dalle telecamere esistenti in un impianto, e i messaggi, inviati dai sensori che delimitano un'area di sicurezza, nella piattaforma cloud di \textbf{Amazon Web Services (AWS)}, sfruttando i servizi gestiti offerti dal provider. 

%contesto
%tecnologie

%sistema
%risultati