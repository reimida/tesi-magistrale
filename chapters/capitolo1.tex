\chapter{Introduzione}


La sicurezza sul lavoro rappresenta un elemento fondamentale all'interno dell' industria manifatturiera, dove l'interazione tra macchinari complessi e operai espone a numerosi rischi. Come noto, gli infortuni sul lavoro nel settore manifatturiero sono tra i più frequenti e gravi, con conseguenze significative sia per i lavoratori che per le aziende. Garantire un ambiente di lavoro sicuro non solo tutela la salute e il benessere dei dipendenti, ma contribuisce anche a migliorare la produttività e a ridurre i costi associati agli incidenti. Essi infatti possono comportare gravi conseguenze per i lavoratori, inclusi infortuni permanenti, invalidità e, in casi estremi, decessi. Tali incidenti non solo influiscono sulla qualità della vita dei dipendenti e delle loro famiglie, ma comportano anche ripercussioni economiche rilevanti per le aziende. I costi diretti includono spese mediche e indennità di infortunio, mentre i costi indiretti comprendono la perdita di produttività, la necessità di sostituzione del personale e i danni alla reputazione aziendale. Oltre alle conseguenze dirette sugli individui, gli incidenti sul lavoro hanno un impatto economico significativo sulle aziende e sulla società nel suo complesso. Le aziende devono affrontare spese legali, aumenti dei premi assicurativi e potenziali sanzioni normative in caso di inadempienza alle leggi sulla sicurezza. Inoltre, la perdita di fiducia dei consumatori e dei partner commerciali può influenzare negativamente le performance finanziarie e la competitività dell'azienda sul mercato. Sul piano sociale, gli incidenti sul lavoro contribuiscono a un aumento dei costi sanitari e riducono la produttività nazionale. La società nel suo complesso subisce un impatto economico derivante dalla perdita di forza lavoro qualificata e dall'aumento delle richieste di assistenza sociale. Pertanto, investire nella sicurezza sul lavoro rappresenta non solo un obbligo etico e legale, ma anche una strategia economica vantaggiosa a lungo termine.


I Dispositivi di Protezione Individuale (DPI) sono strumenti essenziali per prevenire gli incidenti sul lavoro e ridurre l'esposizione dei lavoratori a rischi specifici. DPI comuni includono caschi, guanti, occhiali protettivi, maschere respiratorie e indumenti resistenti agli agenti chimici. L'uso corretto e costante dei DPI è fondamentale per garantire la sicurezza dei lavoratori, ma la loro efficacia dipende dalla conformità e dalla corretta applicazione delle normative da parte dei dipendenti. Inoltre, monitorare l'uso dei DPI in ambienti industriali può risultare complesso, soprattutto in contesti ad alta dinamicità e con elevati volumi di produzione. Tradizionalmente, questo monitoraggio è stato effettuato attraverso ispezioni manuali, che possono essere dispendiose in termini di tempo e risorse e soggette a errori umani. Pertanto, vi è una crescente necessità di soluzioni automatizzate e tecnologicamente avanzate per garantire un controllo efficace e continuo dell'utilizzo dei DPI. L'innovazione tecnologica ha aperto nuove prospettive per migliorare la sicurezza sul lavoro nell' industria manifatturiera. In particolare, la computer vision e il cloud computing emergono come strumenti potenti per automatizzare il rilevamento dei DPI e monitorare in tempo reale le condizioni di sicurezza.

La {\bfseries computer vision} permette alle macchine di interpretare e analizzare immagini e video, identificando automaticamente la presenza e l'uso corretto dei DPI. Attraverso algoritmi di deep learning, i sistemi di computer vision possono riconoscere oggetti specifici, come caschi e guanti, e verificare la loro corretta indossatura da parte dei lavoratori. Questo approccio non solo aumenta l'efficienza del monitoraggio, ma riduce anche la dipendenza da interventi manuali, minimizzando gli errori e garantendo una supervisione costante e accurata. Il {\bfseries cloud computing}, d'altra parte, fornisce l'infrastruttura necessaria per gestire e analizzare grandi quantità di dati provenienti dai sistemi di computer vision. Attraverso piattaforme cloud, è possibile archiviare, elaborare e accedere ai dati in modo scalabile e flessibile, permettendo una gestione centralizzata e accessibile delle informazioni sulla sicurezza. Inoltre, il cloud computing facilita l'integrazione con altri sistemi aziendali, consentendo una visione completa delle operazioni e una risposta tempestiva agli incidenti rilevati. L'integrazione di computer vision e cloud computing rappresenta quindi una svolta nel campo della sicurezza industriale, offrendo soluzioni avanzate per il monitoraggio dei DPI e la prevenzione degli incidenti. 

In questo contesto, la presente tesi si propone di sviluppare un sistema basato su Amazon Rekognition, un servizio di computer vision offerto da Amazon Web Services (AWS), per il rilevamento automatico dei DPI nell' industria manifatturiera. L'obiettivo principale è quello di generare una infrastruttura scalabile per l’analisi di dati semistrutturati e non strutturati all’interno di una fabbrica. In particolare, dato un insieme di macchinari, come ad esempio bracci robotici, si vuole ottenere il controllo dell’effettivo indossamento dei dispositivi di sicurezza da parte degli operatori(operai, manutentori) all'interno di uno stabilimento, in modo tale da garantirne loro la sicurezza sul posto di lavoro.
